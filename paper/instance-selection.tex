\section{Отбор объектов обучающей выборки}
Как правило, работа с выборками большого объёма сопряжена с большими затратами временина обучение модели. Кроме этого, в достаточно популярном методе \(k\) ближайших соседей обучающая выборка хранится полностью, что в случае выборки большого объёма является ограничиващим для него фактором. Таким образом, привлекательными кажутся техники, которые позволяют уменьшить объём обучающей выборки, почти не теряя обобщающей способности. Данная тематика в литературе на английском языке называется instance selection (prototype selection).

В \cite{ps-taxonomy} было проведено крупномасштабное исследование, а также классификация методов отбора объектов обучающей выборки. Авторы также классифицировали их на следующие группы по механизму работы:
\begin{itemize}
    \item методы сгущения (condensation) --- стремятся сократить число точек, далёких от границ классов, в предположении, что они слабо влияют на геометрию границы. Они стремятся сохранить качество классификации на обучающей выборке, при этом качество классификации на тестовой выборке может пострадать. Тем не менее, они, как правило, достигают высокой степени сокращения объёма обучающей выборки;
    \item методы редактирования (edition) --- стремятся сократить число точек, близких к границам классов, несогласованных с соседними, шумовых точек. Данные методы создают более гладкие границы между классами, и приводят к повышению обобщающей способности классификатора, но они в меньшей степени сокращают объём выборки, чем методы предыдущей группы;
    \item гибридные методы (hybrid) --- стремятся найти подмножество выборки как можно меньшей мощности, которое улучшает обобщающую способность на тестовых данных.
\end{itemize}

Столкнувшись со значительными затратами вычислительных ресурсов при тестировании различных методов машинного обучения в нашей задаче, мы решили опробовать несколько методов отбора объектов и оценить, насколько большую пользу они могут принести в данной задаче. В пакете scikit-learn данный класс методов не представлен, а существующие реализации на Python \cite{scikit-protopy} показали себя неудовлетворительно с точки зрения производительности. Поэтому для проведения данной работы некоторые методы, реализованные авторами \cite{ps-taxonomy}, были портированы на язык C. В следующих подразделах выбранные методы отбора объектов описаны и приведены результаты, показанные ими на нашей задаче.

\subsection{Condensed nearest neighbor}
\subsection{Fast condensed nearest neighbor}
\subsection{Class conditioned instance selection}
\subsection{Instance-based 3}
