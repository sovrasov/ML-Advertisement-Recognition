\section{Заключение}
В работе был дан обзор базовых методов машинного обучения и исследована их эффективность применительно к модельной задаче по распознаванию рекламы. На первом этапе произведен подбор основных параметров методов с целью получения оптимального качества классификации на исходных данных. Как и ожидалось, простые методы (
\(k\)NN, LDA) показали не лучшее классификации, но зато они быстро работают. Более продвинутые методы работают медленнее и для них имеет смысл использовать селекцию признаков или эталонов (см. таблицу \ref{table:base-all}). 
\par
