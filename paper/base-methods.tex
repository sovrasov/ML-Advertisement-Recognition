\section{Описание задачи и методологии работы}
\subsection{Задача и набор данных}
Выбранная для тестирования методов задача заключается в определении, является ли элемент видеопотока телевизионной рекламой. Наша группа не занималась получением признакового описания из видеороликов, а лишь проводила эксперименты на готовом наборе данных, полученном исследователями в \cite{vyas}. Авторы указанной работы преобразовали 150 часов видео с 5 новостных каналов (BBC, CNN, CNNIBN, NDTV, TIMESNOW) в признаковые описания коротких видеофрагментов, упорядоченных хронологически. Набор данных включает визуальные (длительность фрагмента, распределение разности кадров, распределение движения, распределение текста на экране и процент движущихся границ объектов на видео) и аудиопризнаки (спектральные характеристики аудиосигнала и bag of audio words). Каждому объекту в наборе данных присвоена метка класса: \(+1\) --- фрагмент является рекламой, \(-1\) --- не является. В таблице~\ref{table:class-distr} указаны соотношения классов в данных каждого из каналов:

\begin{table}
    \centering
    \begin{tabular}{|p{4cm}||c|c|c|c|c|}
    \hline
    Канал & BBC & CNN & CNNIBN & TIMESNOW & NDTV \\ \hline
    \% объектов, соответствующих рекламе & 47.5 & 63.9 & 65.5 & 64 & 73.7\\
    \hline
    \end{tabular}
    \caption{Соотношение классов в наборе данных}
    \label{table:class-distr}
\end{table}

\subsection{Методология}
\subsection{Используемые методы}
\subsubsection{\(k\) ближайших соседей}
\subsubsection{Линейный дискриминантный анализ}
\subsubsection{Машина опорных векторов}
\subsubsection{Случайный лес}
\subsubsection{Градиентный бустинг деревьев решений}

